
Turbulent flows are commonly regarded as good mixers. In fact, its ability to mix may be considered a defining characteristic. To understand the impact of a turbulent flow on mixing, it is useful to examine the length scales present in the concentration field. This is commonly done through characterizing the scalar energy spectrum. The theory of Obukhoff (1949)\cite{Obukhov1949},  Corrsin (1951) \cite{Corrsin1951}, and Batchelor (1959)\cite{Batchelor1959a} captures the characteristics of the scalar energy cascades which have been verified numerically\cite{Eswaran1988,Holzer1994b,Shraiman2000a} and experimentally (CITATION). A key finding from their theory was the emergence of a length scale (now known as the Batchelor scale) where  advection and diffusion balance. The Batchelor scale is given by  $\ell_{b}= \sqrt{\kappa / \Gamma}$ where $\Gamma$ is the typical rate-of-strain of the fluid. For a turbulent flow to be sustained, there must be continual source of energy supplied to the fluid at the kinetic energy dissipation rate per unit mass $\epsilon$. This rate $\epsilon$ is related to the rate of strain $\Gamma = \langle |\nabla \mathbf{u}|^2 \rangle ^{1/2} $ by $\epsilon = \nu \Gamma^2$ where $\nu$ is the fluid viscosity \cite{Doering}. Using this relation, we find that the Batchelor scale is given by $\ell_{b}=\left(\frac{\kappa}{\nu}\right)^{1/2}\left(\nu^3/\epsilon\right)^{1/4}=S^{-\frac{1}{2}}\ell_{k}$ where $S=\frac{\nu}{\kappa}$ is the Schmidt number and  $\ell_{k}=\left(\nu^3/\epsilon\right)^{1/4}$ is the Kolmogorov length \cite{Dimotakis2005,Kolmogorov1941}.  The Batchelor scale defines the cutoff length in the scalar energy spectrum where there is rapid decay past this length according to the Obukhoff-Corrsin-Batchelor theory. Therefore, the presence of the Batchelor scale identifies a limit to the degree of mixing possible by a turbulent flow. 