Mixing is a fundamental fluid mechanism that is crucial to the engineering of industrial processes within the chemical, pharmaceutical, petrochemical, food, and many other industries. Mixing is also important to areas of science including oceanography, turbulence, and atmospheric sciences. An important question to many domains is ``How does one mix efficiently?" We strive to make progress towards this question by studying a series of optimization problems on mixing. 

The first study presented is on optimization of a shell model of mixing. This model is based on a system of ordinary differential equations which mimic the time evolution of the Fourier spectrum of a dye concentration governed by the advection-diffusion equation. We investigate the local-in-time and global-in-time optimization within this model and show that mixing can be limited by diffusion. 

The second study investigates local-in-time optimization of the advection-diffusion partial differential equation. We demonstrate that many of the observations seen in the shell model extend to this setting such as evidence of a limitation on mixing by the inclusion of diffusion.

Lastly, we explore global-in-time optimization of the advection-diffusion equation. This last study is ongoing research at the moment: current results on this topic are presented and a comparison between local-in-time and global-in-time optimization is discussed. 